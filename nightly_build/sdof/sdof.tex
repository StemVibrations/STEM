% ==============================================================================================
\chapter{Single degree of freedom} \label{ch:sdof}
% ==============================================================================================

% ----------------------------------------------------------------------------------------------
\section{Introduction}
% ----------------------------------------------------------------------------------------------

This benchmark compares the STEM numerical solution against the analytical solution, for a single degree of freedom
mass--spring--damper oscillator.
The STEM numerical response is compared with the analytical solution of the linear oscillator subjected to the
sudden application of its self-weight.
The analytical solution is described in~\cite{Verruijt_2010}.

% ----------------------------------------------------------------------------------------------
\section{Model Description}
% ----------------------------------------------------------------------------------------------

% ..............................................................................................
\subsection{Geometry and loading}
% ..............................................................................................
The model consists of a mass--spring--damper sytem, where the damper is connected in parallel to the spring.
The system is subjected to a step load equal to the self-weight of the mass, applied at \qty{0}{\second}.
Figure~\ref{fig:sdof_model} illustrates the configuration of the single degree of freedom oscillator.

\begin{figure}
    \centering
    \includegraphics[width=0.75\textwidth]{sdof/sdof.pdf}
    \caption{Geometry of the single degree of freedom oscillator.}
    \label{fig:sdof_model}
\end{figure}

% ..............................................................................................
\subsection{Materials and numerical parameters}
% ..............................................................................................
The spring--damper element follows a linear elastic constitutive law. The mass is
introduced through a nodal concentrated element. The parameters employed in the calculation are:

\begin{itemize}[noitemsep,topsep=0pt,parsep=0pt,partopsep=0pt]
	\item Spring stiffness: \qty{10}{\kilo\newton\per\meter},
	\item Damping coefficient: \qty{100}{\newton\second\per\meter},
	\item Concentrated mass: \qty{10}{\kilogram},
	\item Gravity acceleration: \qty{9.81}{\meter\per\second\squared}.
\end{itemize}

The dynamic analysis has the duration of \qty{1.0}{\second} with a time step of \qty{0.001}{\second}.
The equations of motion are integrated with the Newmark scheme~\cite{Newmark_1959} using the average
acceleration parameters $\beta = 0.25$ and $\gamma = 0.5$. A

% ----------------------------------------------------------------------------------------------
\section{Results}
% ----------------------------------------------------------------------------------------------
Figure~\ref{fig:sdof_results} shows the vertical displacement history of the mass obtained with STEM and compares
it against the analytical solution. The results show a perfect agreement between the numerical and analytical results.

\begin{figure}[h]
	\centering
	\includegraphics[width=0.8\textwidth]{sdof/time_history.pdf}
	\caption{Comparison between the STEM and analytical vertical for the displacement of the single degree of freedom.}
	\label{fig:sdof_results}
\end{figure}


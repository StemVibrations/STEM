% ==============================================================================================
\chapter{Introduction}
% ==============================================================================================

This document provides a validation overview for the STEM (Soil and Track System Modeling Tool) software,
a Python-based numerical model designed to simulate railway-induced vibrations using the Finite Element Method (FEM).
The objective is to validate numerical accuracy through comparison against analytical solutions and
established benchmark results.

STEM is an open-source Python toolkit to model, simulate, and analyse railway-induced vibrations in
soil-track systems. It builds finite element models, generates meshes, writes Kratos Multiphysics input files,
runs multi-stage geotechnical and structural analyses, and collects results for verification and visualization.

STEM is an open and transparent model that allows users to model railway induced vibrations and test the
effect of mitigation techniques.

Besides the tests presented in this report, STEM is continuously tested through an extensive suite of automated tests.
All the tests are run on Windows and Linux systems on multiple cores, ensuring consistent cross-platform behaviour,
reliable concurrency performance, and early detection of environment-specific issues.
The tests include:

\begin{itemize}[noitemsep,topsep=0pt,parsep=0pt,partopsep=0pt]
    \item \textbf{Unit Tests}: Verify the correctness of individual functions and classes (e.g., mesh generation,
        load application).
    \item \textbf{Benchmarks}: Validate the integrated system against known physical phenomena, analytical solutions,
        or established engineering scenarios.
    % \item \textbf{Validation}: Verify the results of STEM against experimental results.
    \item \textbf{Extensive test suite} available in Kratos Multiphysics, in which 2401 unit tests and 1435 benchmark
        tests are run.
\end{itemize}

In the following chapters, we present a series of benchmark tests that assess the numerical accuracy of STEM.
Each chapter corresponds to a benchmark configuration, including the mathematical model, the numerical approach,
and the result comparison.
